\section{Cauchy-Schwarz Inequality}

	\subsection{Theorem}
		\begin{align}
			\forall u, v \in \mathbb{R}^n |u \cdot v| \leq \|u\| \cdot \|v\|
		\end{align}
  
	\subsection{Proof}
	
		Let $tu + v$ be a family of vectors where t is in $\mathbb{R}$.
	
		\begin{align}
			\forall w \in \mathbb{R}^n 0 \leq w \cdot w \\
			0 \leq (tu + v) \cdot (tu + v) \\
			0 \leq tu \cdot v  + 2u \cdot v + v \cdot tu + v \cdot v \\
			\forall a = u \cdot u, b = 2u \cdot v: 0 \leq t^2
		\end{align}
		
	//TODO CJB REREAD THE FIRST PAGE (having trouble in low light)
		
\section{Triangle Inequality}

	\subsection{Definition}
	
		\begin{align}
			\|u+v\| \leq \|u\| + \|v\|
		\end{align}
		
	\subsection{Proof}
	
		We show that $\|u+v\|^2 \leq (\|u\| + \|v\|)^2$.
		
		\begin{align}
			\|u+v\|^2 &= (u+v) \cdot (u+v) = u \cdot u + 2u \cdot v + v \dot v \\
				&= \|u\|^2 + 2u \cdot v + \|v\|^2 \\
				&\leq \|u\|^2 + 2\|u \cdot v \| + \|v\|^2  \\
				&\leq \|u\|^2 + 2\|u|\ \|v\| + \|v\|^2 & \textrm{by Cauchy} \\
				&\leq (\|u\| + \|v\|)^2
		\end{align}
		
\section{Complex Numbers}

	\subsection{Introduction}
	
		Complex numbers introduce a unit $i$ such that $i^2 = -1$ and generate a set of numbers called $\mathbb{C}$.
		Usually complex numbers are represented as some combination $a + ib$ where $a, b \in \mathbb{R}$.
		
		In complex space you can solve any polynomial equation and the number of solution is the degree \footnote{No matter what people told me I found it hard to understand the importance of complex numbers, but it turns out they become really important in numerical analysis.  In stuff like predator-prey models, oscillations in population dynamics are results of complex eigenvalues.  You can determine whether or not the oscillations of such a population are stable by checking if the imaginary part of the complex values are positive or negative.  In black holes, everything spirals into the black hole along a circle, which can only mathematically be described using complex numbers.  It's surprisingly cool... gets a lot more cool with the stability of orbits for satellites.  You want a satellite to have a slightly unstable orbit so that it doesn't take much energy to shift the position of the satellite. -CJB}.
		In complex space //TODO return to 1/0 section.
		
		A complex number is a pair of real numbers with operations "+" and "*."
		The following demonstrate how complex numbers operate.
		\begin{align}
			(a,b) + (c,d) = (a+c, b+d) \\
			(a,b) * (c,d) = (ac - bd, ad + bc) \\
			(0,1) * (0,1) = (-1,0) = -(1,0) \implies i^2 = 1 \\
			a(1,0) + b(0,1) = (a,b) = a + ib \\
		\end{align}
		
		
		The following demonstrates the logic of complex numbers.
		\begin{align}
			(a+ib)(c+id) &=ac + aid + ibc + (ib)(id) \\
					&= ac + i(ad + bc) + i^2bd \\
					&= (ac - bd) + i(ad+bc)
		\end{align}
		
	\subsection{Definition}
	
		Formally:
		\begin{align}
			\forall a,b \in \mathbb{R}: z = a+ib \\
			a = realpart[z] = [Re(z)]\\
			b = imaginarypart[z] = [Im(z)]
		\end{align}
		
\section{Complex Conjugation}

	The complex conjugate of some complex number z is given by $\overline{z} = a - ib$.
	
	\subsection{Definitions}
	
		\begin{itemize}
			\item $z = a + ib, \overline{z} = (a - ib)$
			\item $z*\overline{z} = (a+ib)(a-ib) = a^2 + b^2$
			\item $|z| = \sqrt{a^2 + b^2} = sqrt{z*\overline{z}}$
		\end{itemize}
		
	\subsection{Theorems}
	
		\begin{itemize}
			\item $\forall z*\overline{z}: z*\overline{z} \in R$
			\item $\forall z*\overline{z}: z*\overline{z} \geq 0$
			\item $\forall z \in \mathbb{C}, z \neq 0: \exists w \in \mathbb{C} \mid zw =1$ \\
					Proof:
					\begin{align}
						&& z = a + ib, w = \dfrac{a-ib}{a^2 + b^2} \\
						zw &= (a + ib)\dfrac{a-ib}{a^2 + b^2} = \dfrac{a^2 - aib + aib - iib^2}{a^2 + b^2} \\
							&= \dfrac{a^2 + b^2}{a^2 + b^2} = 1
					\end{align}
			\item $|z_1z_2| = |z_1||z_2|$\\
					Proof:
					$|z_1z_2| = (z_1z_2)(\overline{z_1z_2}) = z_1z_2\overline{z_1}\overline{z_2} = z_1\overline{z_1}z_2\overline{z_2} = |z_1||z_2|$
		\end{itemize}
	
	
	\subsection{Visualisation}
		
		In the complex plane, complex conjugate is a reflection in the x-axis.
		//TODO generate figures for description of complex conjugate as a reflection of the complex number
		
\section{Complex Spaces}

	$\mathbb{C}^n$ is the complex n space that consists of n tuples of complex numbers.
	Numbers in $\mathbb{C}^n$ can be added, scaled, and otherwise manipulated much like $\mathbb{R}^n$.
	
	\subsection{Definitions}
	 
	 	\subsubsection{Dot Product in $\mathbb{C}^n$}
	 	
		\begin{align}
			(z_1, \ldots, z_2) \cdot (w_1, \ldots, w_n) = z_1\overline{w_1} + \ldots + z_n\overline{w_n}
		\end{align}
		
		This agrees with the dot product on $\mathbb{R}^n$, as $\mathbb{R}^n$ is a subset of $\mathbb{C}^n$.  For $\mathbb{R}$, $a + 0i = a - 0i$, so the conjugate makes no difference.
		
	\subsection{Theorems}
	
		\subsubsection{$v \cdot v$ is in $\mathbb{R}^n$, and nonnegative; $v \cdot v = 0 \iff v = 0$}
		If $v$ is $\mathbb{C}^n$ then $v \cdot v$ is in $\mathbb{R}^n$, and nonnegative; $v \cdot v = 0 \iff v = 0$
				\begin{align}
					v = (z_1, \ldots, z_2) \\
					v \cdot v = z_1\overline{z_1} + \ldots + z_n\overline{z_n} \\
					z_k = a_k + ib_k \\
					z_k\overline{z_k} = a_k^2 + b_k^2 \\
					v \cdot v = a_1^2 + b_1^2 + \ldots + a_n^2 + b_n^2
				\end{align}
				
		This implies $v \cdot v$ is a non-negative i and 0 only if it equals 0 and it equals 0 only if $a_k = 0$ and $b_k = 0$ for all k between 1 and n.
		In this case, v is 0.
		
\section{Matrices as Arrays}

	We can look at matrices as arrays, where a matrix is an array with n rows each with m elements.
	
	The shape of a matrix is defined by n and m.
	If two matrices have the same shape they can be added, where each element of the matrices are added to the corresponding elements of the other matrix.
	
	Scaling a matrix consists of multiplying each element by some scalar.
	
	Multiplying a matrix consists of multiplying each row by a corresponding column in the other matrix.
	It's a little more complicated but presumably people are familiar with this from maths 133.
	
	The transposed matrix consists of a matrix where each row and column index has been switched.
	
	The matrix power is a matrix multiplied by itself some $n$ times.
	
	\subsection{Matrix Multiplication}
	
		\begin{align}
			(AB)C = A(BC) & \textrm{associative} \\
			A(B+C) = AB + AC & \textrm{distributive} \\
			(B+C)A = BA + CA \\
			AB \neq BA
		\end{align}
		
	\subsection{Definitions}
	
			\begin{description}
				\item[square matrix] An n by n matrix.
				\item[trace] The trace of a square matrix $A = [a_{ij}]$ is given by $tr(A) = \sum_{i=1}^{n} a_{ii}$.
				\item[identity matrix] The n by m identity matrix is the $I_n = [a_{ij}]$ where $\forall i = j, a_{ij} = 1 \wedge \forall i \neq j, a_{ij} = 0$
				\item[Matrix Power] 
					$\forall square_matrix[A]\ \ A^0 = I \\
					\forall invertible[A]\ \ A^{-1} = inverse[A]$
				\item[Block Matrix] A matrix partitioned in submatrices.
				\item[Linear Equation]  A linear equation with unknowns is an expression of the form $a_1x_1 + \ldots + a_nx_n = b$ where some vector $a_n$ and the value $b$ are in some number system.
				A solution vector $x_n$ exists that satisfies this equation.
			\end{description}
			
	\subsection{Theorems}
		
		\subsubsection{$A_{mn}I_n = A,\ \ I_nB_{nm} = B$}
		
			\begin{align}
				\sum_{k=1}^{n} a_{ik}I_{ik} = a_{ij}
			\end{align}
			
		\subsubsection{$A_{ik}B_{kj}$}
			
			\begin{align}
				\forall A_{ik}B_{kj} \\
				AB = [C_{ij}] \\
				C_{ij} = \sum_{k = i}^{n} A_{ik}B_{kj}
			\end{align}
		
			
	\subsection{Note on Excluded Material}
	
		I have not included some theorems on addition and multiplication and the like.
		I also have not typeset notes on how to do proofs.
		The assignments seem to do an excellent job of covering this material.
		Should I find the time (probably will), these will be typeset after the midterm to complete the notes properly.